\documentclass[a4paper]{article}

\usepackage[utf8]{inputenc}
\usepackage[brazil]{babel}
\usepackage{graphicx}

\title{Aula latex ce083}
\author{Fernando Mayer}
\date{14 de novembro, 2017}

\begin{document}

\maketitle
\tableofcontents

\section{Introdução}

O LaTeX é um conjunto de {\bf macros} para o processador de textos TeX, utilizado amplamente para a produção de \textbf{textos matemáticos} e \textit{científicos} devido à sua alta qualidade tipográfica. Entretanto, também é utilizado para produção
de cartas pessoais, artigos e {\Large livros} sobre assuntos diversos.

Nova parágrafo.

E outro.

\begin{center}
Texto centralizado.
\end{center}

O Word não possui nem $\sim$ 10\% da capacidade de editoração do LaTeX!

\section{Metodologia}

\subsection{Dados}

Os dados foram coletados da seguinte forma:

\begin{itemize}
\item Forma 1
\item Forma 2
\end{itemize}

Os dados foram importados com:

\begin{verbatim}
dados <- read.table("pib_gapminder.csv", header = TRUE,
                    sep = ",", dec = ".")
\end{verbatim}

\subsection{Análise}

As análises foram:

\begin{enumerate}
\item Análise 1
\begin{enumerate}
\item Sub analise 1
\item Sub analise 2
\end{enumerate}
\item Análise 2
\end{enumerate}

\section{Resultados}

A relação entre expvida e pibpercap pode ser vista na figura \ref{fig:plot}.

\begin{figure}[h]
\centering
\includegraphics[width=0.8\textwidth]{figura1.pdf}
\caption{Legenda da figura.}
\label{fig:plot}
\end{figure}

\section{Discussão}




\end{document}
